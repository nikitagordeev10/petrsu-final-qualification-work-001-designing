\chapter*{Заключение}
\addcontentsline{toc}{chapter}{Заключение}
\vspace{-1cm}
В ходе выполнения работы были достигнуты следующие результаты:

\begin{enumerate}
\item
Описана структура персонального пространства пользователя и классы знаний, содержащиеся в нём.

\item
На основе структуры персонального пространства построена онтологическая модель блогосферы, используемая в приложении SmartScribo, которая позволяет описывать персональные и контекстные данные пользователя, а также данные блогосферы.

\item
Изучена спецификация FOAF на основе которой базируется онтология SmartScribo.

\item
Разработан механизм нотификаций, позволяющий координировать работу агентов приложения и управлять онтологическими данными.

\item
Реализован блог-процессор взаимодействующий с блог-сервисом LiveJournal.

\item
Предложена схема интеграции функций блоггинга SmartScribo в систему SC, что позволяет проводить обсуждение выступлений на конференции.

\item
Предложен способ использования данных интеллектуального туристического приложения M3-Weather в SmartScribo.

\end{enumerate}
Результаты работы опубликованы в статьях \cite{scblogging, smartscribo, smartscribo-old}.

В дальнейшем планируется доработка блог-процессора, а также разработка агентов-медиаторов, выполняющих обработку онтологических данных и предоставляющих расширенные возможности блоггинга для пользователя. Для реализации агентов-медиаторов потребуется расширение модели нотификаций и добавление в неё проактивных нотификаций, используемых в интеллектуальных сценариях блоггинга. 
