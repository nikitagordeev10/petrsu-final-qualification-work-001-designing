\chapter*{Введение}
\addcontentsline{toc}{chapter}{Введение}

В настоящее время мобильные устройства приобрели широкое распространение, что обуславливает актуальность разработки программного обеспечения (ПО). Несмотря на высокую производительность современных мобильных устройств, существует ряд задач, для которых ресурсов такого устройства будет недостаточно. Одним из вариантов решения данной проблемы является применение концепции интеллектуальных пространств (ИП, от англ. Smart Spaces) \cite{oliver:dynamic}.

Концепция ИП является парадигмой программирования, в которой приложение реализуется не монолитным, а распределённым, состоящим из нескольких программ, называемых агентами или процессорами знаний (knowledge processor, KP). Распределённая архитектура описывает, из каких агентов состоит приложение и как осуществляется взаимодействие между ними. Агенты
одного приложения работают достаточно независимо друг от друга и могут быть запущены на различных устройствах.
Программная платформа Smart-M3 \cite{honkola:smart-m3} представляет собой реализацию идеи ИП и позволяет создавать такого рода приложения. 

Взаимодействие между агентами в платформе Smart-M3 осуществляется через брокер семантической информации (Semantic Information Broker, SIB). Обмен данными осуществляется при помощи механизма публикации и подписки \cite{baldoni:evolution}. Суть механизма заключается
в следующем: агенты могут публиковать данные в SIB и могут подписываться на некоторые данные.
Агент, имеющий подписку, получает обновления информации при её добавлении, изменении, или удалении. Причём публикующий KP не имеет никаких сведений об агентах, подписавшихся на его данные. При разработке приложения, использующего парадигму ИП, необходимо учитывать описанный способ взаимодействия между агентами.

Вся информация, хранящаяся в ИП, описывается при помощи модели данных Resource Description Framework (RDF) \cite{rdf}. 
RDF-модель позволяет описывать предметную область и данные, используемые в интеллектуальном приложении.
Формальное описание некоторой области знаний с помощью концептуальной схемы называется онтологией \cite{guarino}. Следовательно, при проектировании приложения, базирующегося на платформе Smart-M3, важным является разработка его распределённой архитектуры, а также онтологии, которая полностью бы описывала предметную область приложения.

На данный момент уже существует ряд приложений, основанных на платформе Smart-M3: сбор контекстных данных на заседаниях~\cite{oliver:context}, интеллектуальный дом~\cite{framling:smart}, поддержка проведения конференций~\cite{scs}, погодное приложение~\cite{m3weather}. 
% в этой области.
%Поэтому исследование, связанное с разработкой таких приложений, а в частности построением онтологий представления знаний в интеллектуальных приложениях, представляется весьма актуальным.

%В связи с ростом популярности социальных сетей актуальным является исследование, связанное с проектированием приложений, взаимодействующих с такими сетями. 
В данной работе будет рассмотрена разработка интеллектуального приложения SmartScribo, позволяющего пользователю работать через мобильное устройство с несколькими блог-сервисами одновременно (мультиблоггинг).
Значительное внимание будет уделено представлению данных блог-сервисов в ИП.
Целью работы является построение онтологии представления знаний блогосферы, а также описание механизмов управления и использования этих знаний.
%Вследствие того, что популярность блогов довольно высока, то приложения, работу с блогами   
Для достижения поставленной цели необходимо решить следующие задачи:
\begin{enumerate}
\item
Определить структуру и классы знаний блогосферы.
\item
Разработать онтологическую модель представления пользовательской информации и блог-данных.
\item
Разработать механизмы управления онтологическими данными, а также синхронизации между агентами.
\item
Разработать агент, взаимодействующий с блог-сервисом LiveJournal.
\item
Предложить способ использования онтологии блогосферы с другими интеллектуальными приложениями.
\end{enumerate}

%Архитектура приложения, использующего парадигму ИП, должна отражать его устройство: расположение всех агентов приложения и порядок взаимодействия между ними.
Онтология интеллектуального приложения должна позволять описывать данные, используемые агентами, и представлять их с помощью RDF модели. Онтология, описывающая предметную область, является важной составляющей интеллектуального приложения, так как благодаря ей агенты имеют представление о том, какие фрагменты данных необходимо обрабатывать.

%Научная новизна работы заключается в том, что впервые было подробно описано и разработано приложение для работы с несколькими блог-сервисами, базирующееся на платформе Smart-M3.

Разработка экспериментального прототипа блог-клиента позволит оценить возможности платформы Smart-M3 при реализации подобного рода приложений. Кроме этого агенты, созданные для данной платформы, могут быть использованы в других приложениях, что позволит разрабатывать новое многофункциональное программное обеспечение без лишних затрат на кодирование.

Данная работа состоит из введения, трёх глав, заключения и списка литературы. В первой главе описывается устройство и архитектура программной платформы Smart-M3, а также принципы взаимодействия между собой интеллектуальных агентов. Во второй главе рассматривается проектирование блог-клиента для ИП и особое внимание уделяется способам представления и управления онтологической информации блогов. В третьей главе приводятся способы интеграции блог-клиента с другими интеллектуальными приложениями. В заключении подводятся итоги проделанной работы, формулируются полученные результаты.
